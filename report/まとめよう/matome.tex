\documentclass[documentclass]{jsarticle}
\usepackage[top=25truemm,bottom=25truemm,left=20truemm,right=20truemm]{geometry}
\usepackage{listings, jlisting, color}
\usepackage[dvipdfmx]{graphicx}
\usepackage{pdfpages}
\usepackage{amsmath}
\usepackage{amssymb, latexsym}
\usepackage{mathtools}
\usepackage{multirow}
\usepackage{color}
\usepackage{ulem}
\usepackage{here}
\usepackage{wrapfig}
\usepackage{tikz}
\usepackage{tcolorbox}
\tcbuselibrary{breakable, skins, theorems}

\usetikzlibrary{intersections, calc, arrows, positioning, arrows.meta}


\newcommand{\Add}[1]{\textcolor{red}{#1}}
\newcommand{\Erase}[1]{\textcolor{red}{\sout{\textcolor{black}{#1}}}}
\newcommand{\ctext}[1]{\raise0.2ex\hbox{\textcircled{\scriptsize{#1}}}}

\lstset{
  basicstyle={\small},
  breaklines=true,
  frame=single,
  tabsize=3,
  numbers=left
}

\begin{document}
\title{UMLに関して}
\author{今村優希}
\maketitle


\newpage

\part{UMLに関して}
\section{UML記述とは}

\section{ユースケース図}

\section{ユースケース記述}

\section{クラス図}

\section{シーケンス図}

\section{アクティビティ図}

\newpage

\part{デザインパターン}
開発者の経験や内的な蓄積をパターンとして形に整理したものが「デザインパターン」.
この部では,そのデザインパターンに関して授業で用いたものを紹介する.

\newpage
\section{composite パターン}
簡単に説明すると,パソコン等で表現されているディレクトリ構造で使用されているパターン.
ディレクトリの中には,ファイルやディジレク鳥が入っており,ディレクトリはそのような「入れ子」になった構造になっており,再帰的な構造を作り出している.

要するに,\textbf{容器と中身を同一視し,再帰的な構造を作る}デザインパターンのことを表す.

\section{observer パターン}


\end{document}