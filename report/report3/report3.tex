\documentclass[documentclass]{jsarticle}
%ディジタル信号処理 Matlab演習1
\usepackage[top=25truemm,bottom=25truemm,left=20truemm,right=20truemm]{geometry}
\usepackage{listings, jlisting, color}
\usepackage[dvipdfmx]{graphicx}
\usepackage{pdfpages}
\usepackage{amsmath}
\usepackage{amssymb, latexsym}
\usepackage{mathtools}
\usepackage{multirow}
\usepackage{color}
\usepackage{ulem}
\usepackage{here}
\usepackage{wrapfig}
\usepackage{tikz}
\usepackage{tcolorbox}
\tcbuselibrary{breakable, skins, theorems}

% 使用する関数の宣言
% (最低限これさえ宣言していれば十分だと思われるものを書いています)
\usetikzlibrary{intersections, calc, arrows, positioning, arrows.meta}


\newcommand{\Add}[1]{\textcolor{red}{#1}}
\newcommand{\Erase}[1]{\textcolor{red}{\sout{\textcolor{black}{#1}}}}
\newcommand{\ctext}[1]{\raise0.2ex\hbox{\textcircled{\scriptsize{#1}}}}

\lstset{
  basicstyle={\small},
  breaklines=true,
  frame=single,
  tabsize=3,
  numbers=left
}

\begin{document}
\title{ソフトウェア設計演習 最終レポート}
\author{222C1021 今村優希}
\maketitle

%\tableofcontents
\clearpage

\newpage

\section{今回のシステム}

今回は下記仕様の「Live Campus」のようなシステムをモデル化する.
システムの大まかな概要を下記に記す.

\begin{tcolorbox}
  \begin{itemize}
    \item 学生は毎学期ごとに、システムから履修登録を行い、許可された講義を受講する。
    履修の可否は、学修細則に基づきシステムが判断する。
    各学期ごとに履修できる科目数には上限があり、同じ時限に重複した科目を履修することはできない。
    \item 学生は、履修登録期間内には何度でも登録、削除、修正(更新)を行うことが出来る。
    \item 学生は、登録した科目と登録可能な科目の時間割をそれぞれ見ることが出来る。
    \item 学生は、自身のこれまでの成績を確認することができる。
    \item 科目には、必修と選択があり、不合格の必修科目があれば履修登録時にシステムが提示し、履修を促す。
    \item 教員は、複数の科目を担当することがあり、システムから各科目の成績を報告する。
    なお、科目担当者は1名とする。
    \item システムは、成績報告期限までに報告していない教員に督促メールを送る。
    なお、成績報告期限の1週間前に成績報告していない教員には、期限を知らせるメールを送るものとする。
  \end{itemize}
\end{tcolorbox}

\section{設計の作成手順}
今回の設計では,分析モデルを作成した後に,設計モデルを作成する手順を取った.

分析モデルでは以下のUML図を作成した.
各UML図が作成し終わると,レビューを行い,後の工程に大きな影響が出ないように工夫を行った.
\begin{itemize}
  \item ユースケース図
  \item ユースケース記述
  \item クラス図
  \item シーケンス図
  \item アクティビティ図
  \item ステートマシン図
\end{itemize}


\newpage

\section{ユースケース図}
システムの概要を把握するためにユースケース図を作成した.
\subsection*{作成}
今回のシステムの内容から,必要なアクターは
\begin{itemize}
  \item 学生
  \item 教員
\end{itemize}
であると考えた.
学生は

\subsection*{レビュー結果メモ}
ユースケース「履修の可否を確認する」に関して,「学生」と関連をつけた方が良いということで,「履修登録する」にincludeで関連付けた.
また,ユースケース「ログインする」が必要だと思ったので,作成し,他のユースケースと関連をつけた.

概要から「履修修正」という要件があったが,今回のシステムでは履修削除と履修登録を組み合わせて行うと考えた.

\subsection*{概要}
レビュー等を通して作成したユースケース図が図\ref*{fig:3-1}である.
%ユースケース図の作成結果
\begin{figure}[H]
  \begin{center}
    \includegraphics*[scale=0.5]{figure/3-1.png}
  \end{center}
  \caption{ユースケース図}
  \label{fig:3-1}
\end{figure}

\newpage

\section{ユースケース記述}
上記で作成したユースケースすべてに対してユースケース記述の作成を行った.
\subsection*{概要と作成結果}

\subsubsection*{1.学生用にログインする}

%1のユースケース記述
\begin{figure}[H]
  \begin{center}
    \includegraphics*[scale=0.6]{figure/4-1.png}
  \end{center}
  \caption{ユースケース図}
  \label{fig:4-1}
\end{figure}

\subsubsection*{2.履修登録する}
\begin{figure}[H]
  \begin{center}
    \includegraphics*[scale=0.6]{figure/4-2.png}
  \end{center}
  \caption{ユースケース図}
  \label{fig:4-2}
\end{figure}

\subsubsection*{3.履修削除する}
\begin{figure}[H]
  \begin{center}
    \includegraphics*[scale=0.6]{figure/4-3.png}
  \end{center}
  \caption{ユースケース図}
  \label{fig:4-3}
\end{figure}

\subsubsection*{4.履修修正する}
\begin{figure}[H]
  \begin{center}
    \includegraphics*[scale=0.6]{figure/4-4.png}
  \end{center}
  \caption{ユースケース図}
  \label{fig:4-4}
\end{figure}

\subsubsection*{5.時間割を確認する}
\begin{figure}[H]
  \begin{center}
    \includegraphics*[scale=0.6]{figure/4-5.png}
  \end{center}
  \caption{ユースケース図}
  \label{fig:4-5}
\end{figure}

\subsubsection*{6.成績を確認する}
\begin{figure}[H]
  \begin{center}
    \includegraphics*[scale=0.6]{figure/4-6.png}
  \end{center}
  \caption{ユースケース図}
  \label{fig:4-6}
\end{figure}

\subsubsection*{7.未習得の履修を促す}
\begin{figure}[H]
  \begin{center}
    \includegraphics*[scale=0.6]{figure/4-7.png}
  \end{center}
  \caption{ユースケース図}
  \label{fig:4-7}
\end{figure}

\subsubsection*{8.教員用にログインする}
\begin{figure}[H]
  \begin{center}
    \includegraphics*[scale=0.6]{figure/4-8.png}
  \end{center}
  \caption{ユースケース図}
  \label{fig:4-8}
\end{figure}

\subsubsection*{9.成績の報告をする}
\begin{figure}[H]
  \begin{center}
    \includegraphics*[scale=0.6]{figure/4-9.png}
  \end{center}
  \caption{ユースケース図}
  \label{fig:4-9}
\end{figure}

\subsubsection*{10.成績報告督促を送る}
\begin{figure}[H]
  \begin{center}
    \includegraphics*[scale=0.6]{figure/4-10.png}
  \end{center}
  \caption{ユースケース図}
  \label{fig:4-10}
\end{figure}

\subsubsection*{11.成績報告期限メールを送る}
\begin{figure}[H]
  \begin{center}
    \includegraphics*[scale=0.6]{figure/4-11.png}
  \end{center}
  \caption{ユースケース図}
  \label{fig:4-11}
\end{figure}


\subsection*{レビュー}


\newpage

\section{クラス図}
\subsection*{概要}


\subsection*{作成したもの}


%クラス図の表示


\newpage

\section{シーケンス図}
\subsection*{概要}
作成したシーケンス図は乗降用,チケット購入用,のシーケンス図を作成した.
基本的にユースケース記述に記載したことをそのまま実現した.
これらの図にはクラス図で出てきたもののみを用いて実現するよう工夫した.
要するに,新しいクラスを作るような作業は行わなかった.

\subsection*{レビュー結果メモ}
\begin{itemize}
  \item ユースケース記述との相違点があったので,順番を入れ替える等の作業を行った
\end{itemize}
\subsection*{作成した図}


\newpage

\section{アクティビティ図}
\subsection*{概要}
作成したアクティビティ図は「バスの位置を確認する」と「バスの位置を記録する」である.


\subsection*{作成した図}


\newpage

\section{ステートマシン図}
\subsection*{概要}


\subsection*{レビュー}


\section{まとめ}


\end{document}